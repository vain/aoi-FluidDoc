% Basics
\documentclass[12pt,a4paper]{scrartcl}
\usepackage[ngerman]{babel}
\usepackage[utf8x]{inputenc}

% Grundlegende Pakete
\usepackage{amsmath,amsfonts,amssymb}
\usepackage{graphicx}
\usepackage{color}
\usepackage{hyperref}

% Einfache Anpassungen und PDF-Metadaten
\parindent 0em
\parskip 1em
\definecolor{darkblue}{rgb}{0,0,0.5}
\hypersetup{
	pdftitle={Fluid Simulation 0.6.6},
	pdfauthor={TroY},
	pdfkeywords={AoI, Physics, Fluid},
	%
	colorlinks=true,
	breaklinks=true,
	linkcolor=black,
	menucolor=darkblue,
	urlcolor=darkblue
}


% Meine sonstigen Sachen
% ####################################################################
% ### SHORTCUTS

\newcommand{\lb}{ \ensuremath{ \left(  } }
\newcommand{\rb}{ \ensuremath{ \right) } }

\newcommand{\ma}{ \ensuremath{ \begin{matrix} } }
\newcommand{\me}{ \ensuremath{ \end{matrix}   } }

\newcommand{\kma}{ \ensuremath{ \begin{pmatrix} } }
\newcommand{\kme}{ \ensuremath{ \end{pmatrix}   } }

\newcommand{\aleq}[1]{\begin{align*}#1\end{align*}}

\newcommand{\suseq}{\ensuremath{\subseteq}}
\newcommand{\suse}{\ensuremath{\subset}}

\newcommand{\Bild}{\ensuremath{\tx{Bild }}}
\newcommand{\Kern}{\ensuremath{\tx{Kern }}}
\newcommand{\Rang}{\ensuremath{\tx{rang }}}
\newcommand{\Grad}{\ensuremath{\tx{grad }}}
\newcommand{\End}{\ensuremath{\tx{End }}}
\newcommand{\Sign}{\ensuremath{\tx{sgn }}}

\newcommand{\supseq}{\ensuremath{\supseteq}}
\newcommand{\supse}{\ensuremath{\supset}}

\newcommand{\setN}{\ensuremath{\mathbb{N}}}
\newcommand{\setZ}{\ensuremath{\mathbb{Z}}}
\newcommand{\setQ}{\ensuremath{\mathbb{Q}}}
\newcommand{\setR}{\ensuremath{\mathbb{R}}}
\newcommand{\setC}{\ensuremath{\mathbb{C}}}

%\newcommand{\qed}{\hfill \ensuremath{\Box}}

\newcommand{\RA}{\ensuremath{\Rightarrow}}
\newcommand{\LA}{\ensuremath{\Leftarrow}}
\newcommand{\ra}{\ensuremath{\rightarrow}}
\newcommand{\la}{\ensuremath{\leftarrow}}

\newcommand{\q}{\ensuremath{\: \:}}
\newcommand{\qq}{\ensuremath{\quad}}

\newcommand{\ub}[2]{\ensuremath{\underbrace{#1}_{#2}}}

\newcommand{\ul}[1]{\underline{#1}}
\newcommand{\ol}[1]{\overline{#1}}
\newcommand{\tx}[1]{\ensuremath{\text{#1}}}


\newcommand{\enumA}{ \begin{enumerate} }
\newcommand{\enumE}{ \end{enumerate}   }

\newcommand{\itA}{ \begin{itemize} }
\newcommand{\itE}{ \end{itemize}   }

\newcommand{\mlcA}{ \tiny \ma }
\newcommand{\mlcE}{ \me \normalsize }

\newcommand{\ad}[1]{\ensuremath{\mathfrak{#1}}}

\newcommand{\hr}{ \rule{\textwidth}{0.4pt} }

\newcommand{\VORBEI}[1]{\hfill \textit{Ende Vorlesung #1} \bigskip}

% ####################################################################
% ### LISTINGS
\definecolor{lststringcolor}{rgb}{0.20,0.50,0.20}
\definecolor{lstcommentcolor}{rgb}{0.40,0.40,0.40}
\definecolor{lstkeywordcolor}{rgb}{0.50,0.10,0.10}
\definecolor{lstidcolor}{rgb}{0.10,0.10,0.50}
\definecolor{lstbg}{rgb}{0.97,0.97,0.97}
\usepackage{listings}
\lstset{
	language=Java,
	basicstyle=\ttfamily\tiny,
	numberstyle=\ttfamily\tiny,
	keywordstyle={\color{lstkeywordcolor}\normalfont\bfseries},
	commentstyle={\color{lstcommentcolor}\itshape},
	%stringstyle={\color{lststringcolor}\underbar},
	stringstyle=\color{lststringcolor},
	%identifierstyle=\color{lstidcolor},
	backgroundcolor=\color{lstbg},
	numbers=left,
	xleftmargin=2em,
	stepnumber=1,
	tabsize=2,
	breaklines=true,
	breakatwhitespace=true,
	showspaces=false,
	showstringspaces=false,
	showtabs=false,
	frame=single
}

\lstdefinestyle{Java}
{
	language=Java
}

\lstdefinestyle{Python}
{
	language=Python
}

\lstdefinestyle{TeX}
{
	language=TeX
}

% Headers
\usepackage[automark]{scrpage2}
\pagestyle{scrheadings}

\setheadsepline{1pt}
%\setfootsepline{1pt}

\ohead{\thepage}
\chead{}
\ihead{\headmark}
\cfoot{}


% Im Header keine Nummerierung der Sections ...
\renewcommand{\sectionmarkformat}{}
\renewcommand{\subsectionmarkformat}{}
% ... und im Text auch nicht
\renewcommand*{\othersectionlevelsformat}[3]{}


% Befehle nur für dieses Dokument
\usepackage{xspace}

\newcommand{\aoi}{Art of Illusion\xspace}
\newcommand{\blender}{Blender\xspace}
\newcommand{\deltor}{\texttt{delt0r}\xspace}
\newcommand{\fluidsim}{Fluid-Simulation\xspace}

\newcommand{\addpic}[1]{\begin{center}\textbf{BILD HINZUFÜGEN: #1}\end{center}}

\usepackage[font={footnotesize},labelfont={footnotesize,bf},labelsep=endash]{caption}
\newcommand{\fpic}[3]
{
	\begin{figure}[htb]
	\centering
	\includegraphics[width=0.5\textwidth]{#1}
	\caption{#3\label{#2}}
	\end{figure}
}

\newcommand{\fachb}[1]{\textit{#1}}
\newcommand{\rand}[1]{\marginline{\tiny{\texttt{#1}}}}



\begin{document}

% Titelseite mit Copyleft
\author{TroY \\ \url{http://www.uninformativ.de}}
\title{Fluid Simulation 0.6.6 \\ mit Art of Illusion}
\date{20. April 2009}
\maketitle
\thispagestyle{empty}

\vspace{100mm}

\begin{center}
	Dieses Dokument steht unter der \\
	\textit{Creative Commons Attribution-Noncommercial-Share Alike 3.0 Germany License}, \\
	\url{http://creativecommons.org/licenses/by-nc-sa/3.0/de/}.

	\includegraphics[scale=0.5]{cc.pdf} \\
	\url{http://creativecommons.org/}
\end{center}

\pagebreak


% Inhaltsverzeichnis
\tableofcontents
\thispagestyle{empty}
\pagebreak


% Dokument

\section{Einleitung}
Bereits seit 2005 arbeitet \deltor an einem Plugin zur Physiksimulation.
Zu Beginn sollte es ein reines Plugin für \blender werden. Er entdeckte
dann zwar recht bald, dass in dieser Hinsicht bereits andere
Programmierer an der Arbeit waren, entschied sich aber trotzdem dafür,
sein eigenes Plugin weiterzuentwickeln - zum Vorteil für die Nutzer von
\aoi. Denn auf unser schönes Programm stieß \deltor erst Anfang 2007.
Der Kern seiner Simulation war seit jeher in Java geschrieben, was der
Integration mit \aoi natürlich sehr entgegen kam. So nahmen die Dinge
dann ihren Lauf.

Sein Plugin ist zwar immernoch in einer recht frühen Entwicklungsphase,
aber es ist durchaus schon zu gebrauchen - und sei es nur, um Spaß zu
haben. Die jeweils aktuelle Version ist in seinem Blog verlinkt:

\url{http://delt0r.blogspot.com/}

Da es an Dokumentation jedoch ein bisschen mangelt, die Informationen in
Foren verstreut sind und nicht jeder die Zeit hat, sich mittels "<Trial
and Error"> einzuarbeiten, entstand dieser Guide, um den Einstieg etwas
zu erleichtern. Eigentlich ist das schon die zweite Anleitung dieser
Art, denn bereits von eineinhalb Jahren habe ich eine solche
geschrieben:

\url{http://www.uninformativ.de/?section=news&ndo=single&newsid=22}

Bis vor kurzem war sie noch zu gebrauchen, aber Anfang April 2009 gab es
dann einen etwas größeren Sprung in der Entwicklung, sodass spätestens
jetzt eine Aktualisierung notwendig wurde.

\subsection{Installation}
Die Installation gestaltet sich denkbar einfach: Man besucht \deltor's
Blog, sucht den Link und lädt eine Datei namens \texttt{Physics.jar}
herunter. Diese Datei muss dann in den Unterordner \texttt{Plugins} des
\aoi-Verzeichnisses kopiert werden. Fertig.

Um die aktuelle Version nutzen zu können, sollte man am besten \aoi
in der Version 2.7.2 verwenden. Für den Fall, dass \aoi noch gar nicht
installiert ist, kann man es sich unter folgender URL herunterladen:

\url{http://www.artofillusion.org}

\subsection{Funktion und Einsatzzweck der Simulation}
Wie der Name schon vermuten lässt, ist der primäre Zweck, Flüssigkeiten
zu simulieren. Diese entstehen in sogenannten "<Emitters"> und
interagieren dann mit als "<Boundary"> definierten Objekten in der
Szene.\footnote{Dem Plugin fehlt noch eine deutsche Übersetzung, daher
werde ich mich hier an die englischen Begriffe halten, um Verwirrung zu
vermeiden. Sollte der eine oder andere Begriff Probleme bereiten, dann
findet sich hier ein erstklassiges Online-Wörterbuch:
\url{http://dict.leo.org}} Alternativ kann man auch ganze Objekte
als "<Fluid"> vordefinieren, um auf einen Schlag viel Flüssigkeit in die
Szene zu bringen.

Mittlerweile beherrscht das Plugin jedoch auch die Simulation von
sogenannten "<Soft Bodies">. Damit sind Körper gemeint, die im
Wesentlichen deformierbar sind. Man kann damit in gewissem Rahmen zwar
auch feste Körper simulieren, stößt aber bald an die Grenzen. Die Gründe
dafür werden später näher erläutert. Eine richtige Simulation starrer
Körper ("<Rigid Bodies">) wird es erst in der Zukunft geben.

Das Plugin ist dabei auch tatsächlich eine \emph{Simulation}. Leider
steht das in einem gewissen Widerspruch, was das "<Art"> in \aoi
betrifft. Simulationen versuchen, der Realität ähnlich zu sein, was zur
Folge hat, dass man sie nur sehr schwer kontrollieren kann. Es ist
beispielsweise nicht möglich, das Anstoßen zweier Bierkrüge, von denen
der Inhalt des einen in den anderen überschwappt, so darzustellen, wie
man es sich gedanklich vorstellen würde. Stattdessen muss eine Animation
erstellt werden, welche die Bierkrüge im Vorfeld genau so bewegt, dass
die in ihnen enthaltene Flüssigkeit dann auch tatsächlich im richtigen
Moment "<schwappt">.  Das ist nicht einfach, da sich die Flüssigkeit
realitätsgetreu verhält und man die Krüge nicht mit den Händen
\emph{anfassen} kann. Dadurch wird das Einschätzen der wirkenden Kräfte
zu keiner leichten Aufgabe.

Auf der anderen Seite ist aber genau das die Herausforderung.

Abgesehen von den künstlerischen Aspekten bietet die Simulation
natürlich eine großartige Spielwiese. Es macht einfach Spaß, einen Turm
aus Kisten zu bauen und dann zuzusehen, wie dieser durch Wasser
weggespült wird. Oder die Simulation einer Kegelbahn. Oder nachgebildete
Hydraulik, die vielleicht sogar einem kleinen Roboter Leben einhaucht.
Oder die realitätsgetreue Darstellung von Kleidung und Stoffen...

Ein bisschen Zukunftsmusik ist in diesen Beispielen allerdings noch
enthalten: Im Moment beherrscht das Plugin noch keine Reibung.

\pagebreak
\section{Grundlagen}
Partikel = Vertices, PolyBoundaries, Backen

\section{Flüssigkeitssimulation}
einfach mal ein Beispiel durchgehen, möglichst viele Optionen
besprechen.

\section{Simulation deformierbarer Körper}
dito... Feder-Masse-System und so

\section{Texturieren und Rendern}
Raster-Renderer empfehlen, Grid für den Raytracer


\end{document}

% vim: set tw=72 :
